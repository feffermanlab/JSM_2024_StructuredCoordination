\documentclass{beamer}
\usetheme{berlin}
\definecolor{utorange}{HTML}{F77F00}
\definecolor{smokey}{HTML}{4C4D4F}
\definecolor{limestone}{HTML}{F0EDE4}
\definecolor{accent}{RGB}{0,018,147}
\definecolor{darkutorange}{HTML}{a95700}
\definecolor{lightutorange}{HTML}{ffe8d0}

\setbeamercolor*{title}{fg=smokey,bg=limestone}
\setbeamercolor*{author}{fg=smokey}
\setbeamercolor*{institute}{fg=smokey}
\setbeamercolor*{date}{fg=smokey}
\setbeamercolor*{frametitle}{fg=smokey, bg=limestone,}
\setbeamercolor*{structure}{fg=utorange}
\setbeamercolor{normal text}{fg=smokey,bg=white}
\setbeamercolor{alerted text}{fg=utorange}
\setbeamercolor{example text}{fg=smokey}
\setbeamercolor{palette primary}{fg=black, bg=utorange}
\setbeamercolor{palette secondary}{fg=white, bg=utorange}
\setbeamercolor{palette tertiary}{fg=white, bg=darkutorange}

\DeclareMathOperator*{\argmax}{\text{argmax}}
\DeclareMathOperator*{\argmin}{\text{argmin}}
\DeclareMathOperator{\uu}{\mathbf{u}}
\usepackage{tikz}
\usetikzlibrary{calc, shapes, fit}

\title{A Discrete Structured Coordination Game and its Applications in Theoretical Ecology}
\author{John McAlister}
\institute[Fefferman Lab]{Univeristy of Tennessee - Knoxville}
\begin{document}
	\begin{frame}[plain]
		\centering
		\maketitle
		\includegraphics[height=3cm]{images/LogoCenter.jpg}
	\end{frame}
\section{Introduction}
\begin{frame}{The Coordination Game}
	The Coordination game, at its most basic, is a two player game with the following payoff matrix
	\begin{center} 
		\begin{tabular}{c|cc}
			&A&B\\
			\hline 
			A&a,a&c,d\\
			B&d,c&b,b
		\end{tabular}
	\end{center}
	
	With the assumption that $a>d$ and $b>c$. It has two Pure Strategy Nash equilibria, $(A,A), (B,B)$, and a mixed strategy Nash equilibrium where both players play strategy $A$ with probability $p=\frac{b-d}{a+b-c-d}$.
\end{frame}
\begin{frame}{The Coordination Game}
	The game can also be considered among many players where pairs of players are selected to play uniformly randomly.
	\begin{itemize}
		\item Using myopic best response as a replicator dynamic, A group playing this game with \textit{high inertia} and $\varepsilon$\textit{-noise} will converge to the Risk Dominant Nash Equilibrium \footnote{Kandori, Michihiro, et al. “Learning, Mutation, and Long Run Equilibria in Games.”}
		
		\item Changing the way in which the pairs are selected can lead to different equilibrium selection \footnote{Robson, Arthur J., and Fernando Vega-Redondo. "Efficient equilibrium selection in evolutionary games with random matching."}
		
		\item Convergence time is very slow \footnote{Ellison, Glenn. "Basins of attraction, long-run stochastic stability, and the speed of step-by-step evolution."}
	\end{itemize}
\end{frame}
\begin{frame}{The Structured Coordination Game}
	If player selection is not uniform random across all players we call this a structured coordination game. 
	\begin{itemize}
		\item Complete Graphs (Identical to unstructured game)
		\item Circular Cities
		\item Square Lattice 
	\end{itemize}
	Each of these previous studies have taken the approach of using symmetry to reduce the state space, considering a transition matrix without noise and establishing a distance between states through mutation. \footnote{Ellison, Glenn. ``Learning Local Interactions and Coordination."}$^,$\footnote{Weidenholzer, Simon. ``Coordination games and local interactions: a survey of the game theoretic literature."}
\end{frame}

\begin{frame}{The Structured Coordination Game}
	Consider this most general setting: For a connected graph $G(V,E)$ each vertex $v\in V$ plays a strategy $c$ from a set of pure strategies $C$. and the payoff for $v$ is given by \begin{equation}
		w(v,c|\uu)=|\{x\in \Gamma (v);\uu_x=c\}|
	\end{equation}
	where $\uu$ is the strategy profile and $\uu_x$ is the strategy that $x$ is using. 
	In this case the game is at its most simple where the Payoff matrix is simply $I_n$. Note, however, we are not limited to only 2 strategies as before. 
\end{frame}

\begin{frame}{The Structured Coordination Game as a Dynamical System}
	In keeping with the previous work, we use a myopic best response as our replicator dynamic. In this way we construct a sequence of strategy profiles $\uu(t)$ with 
	\begin{equation}
		\uu_v(t+1)\in \argmax_{c\in C}\{w(v,c|\uu(t))\}
	\end{equation}
It may be that $|\argmax|>1$ so we break ties in the following way ($\varepsilon$-inertia) 
\begin{itemize}
	\item if $\uu_v(t)\in \argmax_{c\in C}\{w(v,c|\uu(t))\}$ then $ \uu_v(t+1)=\uu_v(t)$.
	\item else, select from $\argmax_{c\in C}\{w(v,c|\uu(t))\}$ uniform randomly. 
	\end{itemize}
\end{frame}

\begin{frame}{Goals and Questions}
	It is clear to see that Equilibria in the dynamical system are Nash Equilibria of the game. There are many questions that arise from this system:
	\begin{itemize}
		\item Can we Characterize Equilibria for a general graph (or a particular topology) and determine stability conditions?
		
		\item Given a set of ``boundary conditions" can we find a strategic interpolation which is a Nash Equilibrium?
		
		\item What might a strategically continuous version of this game look like and what can it tell us about cooperative behaviors in ecology?
	\end{itemize}
\end{frame}
\section{Equilibria and Stability}
\begin{frame}{Equilibria}
	\begin{figure}
		\includegraphics[width=0.4\linewidth]{images/OralFig2v3}
		\includegraphics[width= 0.4\linewidth]{images/OralFig3v4}
		\caption{A trivial and a non-trivial equilibrium in a 4-regular graph}
	\end{figure}
\end{frame}

\begin{frame}
	{Equilibra: Terminology} %%%This will need to change
	In order to discuss these ideas we need a definition
	\begin{block}{\textbf{Clique}}
		A subgraph of $G$ spanned by all of the vertices using a particular strategy. 
	\end{block}  
	and some notation
	\begin{block}{Notation}
		\begin{tabular}{rp{9cm}} 
			$Q(\mathbf{u})$& The set of all cliques in a strategy profile $\mathbf{u}$ \\
			$q^i$& A clique in $Q(\mathbf{u})$ in which vertices use strategy $i$\\
			$\partial q^i$& Vertices in $q_i$ which have neighbors in other cliques\\
		\end{tabular}
	\end{block}
\end{frame} 	
\begin{frame}{Preliminary Results}
	\begin{block}{The only equilibrium in $K_n$ is the trivial equilibrium}
		Suppose there is a equilibrium coloring $\mathbf{u^*}$ with $m\geq2$ cliques, $q^{c_1},...,q^{c_m}$. Because every pair of vertices shares an edge, $w(v,c_i|\uu)=|q_i|$ if $v\notin q^{c_i}$, and $w(v,c_i|\uu)=|q_i|-1$ if $v\in q^{c_i}$.  Consider a vertex $v_1\in q^{c_1}$. Because it is at equilibrium $$w(v_1,c_1|\uu^*)=\argmax_C\{w(v_1,c|\uu^*)\}=:a.$$
		Therefore $|q^{c_1}|=a+1$. Moreover $w(v_1,c_2|\uu^*)\leq a$ so $|q^{c_2}|\leq a$.
		Now consider a vertex  $v_2\in q^{c_2}$. $w(v_2,c_2|\uu^*)=|q^{c_2}|-1\leq a-1$ and $w(v_2,c_1|\uu^*)=a+1$. 
		Thus $\uu^*_2=c_2\notin \argmax\{w(v_2,c|\uu^*)\}$ so $\uu^*$ is not an equilibrium.  
	\end{block}
\end{frame}

\begin{frame}{Preliminary Results}
	\begin{block}{$K_{n,m}$ admits an equilibrium with $d$ cliques iff $d|n$ and $d|m$}
		$\impliedby$ It is easy to construct an equilibrium strategy profile with $d$ cliques
		
		$\implies$ I argue by contradiction. $K_{n,m}=E^n+E^m$  Without loss of generality suppose that $d$ is not a divisor of $m$. Suppose $\uu^*$ is an equilibrium strategy profile with $d>1$ cliques. $d$ does not divide $m$ so $\exists$ strategies $i$ and $j$ such that $|q^i\cap E^m|>|q^j\cap E^m|$. If $\uu^*$ is at equilibrium it must follow that $q^j\cap E^n=\emptyset$. It then follows $q^j\cap E^m =\emptyset$. If $q^j$ is empty then there are not $d$ cliques in $\uu^*$. This contradiction proves the result.   
	\end{block}
\end{frame}

\begin{frame}{General Remarks on Equilibria}
	In general the process of finding equilibria is equivalent to finding this kind of vertex partition.
	
	\begin{equation}
		P=\cup_{i=1}^m P^{c_i}\text{ where } x\in P^{c_i}\Rightarrow |\Gamma(x)\cap P^{c_i}|\geq |\Gamma(x)\cap P^{c_j}|\, \forall i,j 
	\end{equation}
	There are other similar kinds of graph partitions
	\begin{itemize}
		\item Modularity Partitions
		\item Minimum Cut Partitions
	\end{itemize}
	Problems in community detection rely heavily on these partitions but the game theory literature has not relied on insights from this field. 
\end{frame}

\begin{frame}{Barriers to Partitioning}
	In the light of graph partitioning, this question is a special kind of community detection. Without more restrictions on the partition we run into problems: 
	\begin{itemize}
		\item We do not know a priori how many strategies will be present in an equilibrium (equivalently how many parts the graph will be partitioned into)\footnote{Newman, M. E. J. "Modularity and Community Structure in Networks."}
		\item It is not assumed that a non-trivial equilibrium (non-trivial partition) exists
		\item Sub-optimal equilibria are still of interest to us but may be missed by partitioning algorithms.\footnote{Clauset, Aaron, Newman, M. E. J., Moore, Cristopher. "Finding Community Structure in very large networks}  
	\end{itemize}
\end{frame}

\begin{frame}{General Remarks on Equilibria}
	To support what can be learned from the partitioning approach in this context we also want to investigate equilibria through other means
	
	\begin{itemize}
		\item Energy-like estimates: Neither $1-Q$ (modularity) nor the number of edges which connect different cliques are appropriate energies for this context
		\item Extremal Techniques: Describing necessary conditions on clique boundaries by investigating minimal separators \footnote{B\'ela Bollob\'as. ``Extremal Graph Theory"}
	\end{itemize}
\end{frame}


\begin{frame}{Stability}
	When viewed as a dynamical system, stability can be considered in multiple ways 
	\begin{itemize}
		\item Local Stability - A vertex is stable to a single perturbation
		\item Global Stability - Every vertex is stable to a single perturbation
		\item Convergent Stability - Every ``nearby" strategy profile evolves into this strategy profile
		\item Structural Stability - A strategy profile is stable to perturbations of the graph structure.\footnote{Ely, Jeffery C. ``Local Conventions"}$^,$\footnote{Kirkland, S. Li Y. McAlister J. S. Zhang X. ``Edge Addition and the change in Kemeny's Constant"}
	\end{itemize}
\end{frame}

\begin{frame}{Cycles}
	When viewed as a dynamical system we also see the (common) emergence of cycles. We conjecture that $n$-cycles with $n>2$ are impossible. 
	\begin{figure}
	\includegraphics[width =  \linewidth]{images/OralFig4v2}
	\end{figure} 
\end{frame}

\begin{frame}{Goals and Further Questions}
	We hope to 
	\begin{itemize}
		\item Prove the $n$-cycle conjecture,
		\item Find analytically tractable stability criteria in multiple senses
		\item Use energy-like estimates to describe basins of stability for equilibria
		\item Use techniques from extremal graph theory to describe possible equilibria by the structure of the graph by which they are admitted
		\item Summarize with some computational results relating edge density or connectance to clique number 
	\end{itemize}
\end{frame}


\section{Boundary Value Problem}
\begin{frame}{Boundary Value Problem}
	A vital application of this idea is a boundary value conception of the problem.
	\begin{block}{Boundary Value Problem}
		Suppose $B\subset V$ is a subset of vertices which are assigned strategies by $f:B\rightarrow C$. Is there a strategy profile such that $\uu_v=f(v)$ for all $v\in B$ and $\uu$ is an equilibrium strategy profile?
	\end{block}
\end{frame}
\begin{frame}{Intuition}
	Suppose that in  a signaling network $G$
	\begin{itemize}
		\item Each vertex uses exactly one language to send a receive signals. 
		\item A subset of vertices are assigned languages which they cannot change
		\item Translation of the signal is costly relative to transmission. 
	\end{itemize}

	How can we assign a ``language" to each vertex so that each vertex is minimizing their own ``translation" burden?
\end{frame}

\begin{frame}{Initial Thoughts}
	This is rather speculative but the direction forward may look like this:
	\begin{itemize}
		\item Trying to prove existence of such an equilibrium interpolation for an admissible set of boundary values. 
		\item (Almost Equivalently) finding a class of boundary values for which an interpolation can be made
		\item Seeking out an algorithm to build an equilibrium interpolation through graph reductions.  
		\item Seeking out an algorithm to build an equilibrium interpolation through partitioning
	\end{itemize}
\end{frame}
\begin{frame}{Partitioning}
	As before this becomes a partitioning problem but with an added components
	\begin{block}{Boundary Value Problem }
		For a graph $G(V,E)$, $B\subset V$ and $F:B\rightarrow C$, find a partition $\mathcal{P}=\cup_{i=1}^mP^{c_i}$ such that 
		\begin{equation}
			\begin{split}
				i)\quad & |\Gamma(x)\cap P^{\uu_x}|\geq |\Gamma(x)\cap P^{c}|\,\forall x\in V, c\in C \\
				ii)\quad &  x\in P^{f(x)}\,\forall x\in B 
			\end{split}
		\end{equation}
	\end{block}
	In this case, we have extra restrictions on the partition which make partitioning results themselves very helpful. Consider the natural correspondence between partitions and strategy profiles.
\end{frame}

\begin{frame}{Minimum Cut Partitioning}
	Let $A$ be the adjacency matrix for the graph $G$ which has $m$ edges. 
	\begin{equation}
		E(\uu) = \sum_{v,w}A_{vw}(1-\delta(\uu_v\uu_w))= 2m-\sum_{v,w}A_{vw}\delta(u_v,u_w)	\end{equation}
	
	The minimum cut partition into $|C|$ parts is that partition that which minimizes $E$. 
	
	Note that a minimum cut partition requires exactly $|C|$ parts. Otherwise, the minimum cut partition would always be the trivial partition. 
\end{frame}

\begin{frame}{Minimum Cut Partitioning}
	\begin{block}{Every Minimum Cut Partition with parts of size $>1$ corresponds to an equilibrium partition.}
		Suppose that $\mathcal{P_\star}=\cup_{c\in C}P_\star^c$ is a minimum cut partition of $G(V,E)$ corresponding to a strategy profile $\uu^\star$ which is not an equilibrium. Therefore $\exists v\in V$ which is using strategy $r\in C $ which for which $w(v,r|\uu)<w(v,s|\uu)\iff |\Gamma(v_\star)\cap P_\star^{r}|<|\Gamma(v_\star)\cap P_\star^s|$ for some $s\in C$. Let $\hat\uu$ be a new strategy profile where $\hat\uu_v=\uu_v^\star$ for all $v\neq v_\star$ and $\hat\uu_{v_\star}=s$. Call the corresponding partition $\hat{\mathcal{P}}$. We can easily compute that
		\begin{equation}
			E(\uu^\star)-E(\hat\uu) = -2\sum _{v\neq v_\star}A\delta(\uu^\star_v,\uu^\star_{v_\star})+2\sum _{v\neq v_\star}A\delta(\hat\uu_v,\hat\uu_{v_\star})
		\end{equation}
		Which is clearly positive, This contradiction proves the result.    
	\end{block}
\end{frame}

\begin{frame}{Modularity Partitioning}
	\begin{equation}
		Q(\uu) = \frac{1}{2m}\sum_{v,w}\left[A_{vw}-\frac{d_vd_w}{2m}\right]\delta(\uu_v,\uu_w)
	\end{equation}
	let $B=[[A_{ij}-\frac{d_id_j}{2m}]]$ be the modularity matrix let $\chi^{c}(\uu)=[\delta (\uu_i,c)]_{i=1}^n$ then (7) is equivalent to 
	\begin{equation}
		Q(\uu)= \frac{1}{2m}\sum_{c\in C} {\chi^c(\uu)}^TB\chi^c(\uu)
	\end{equation}
	A modularity partition is a partition into $|C|$ parts which maximizes $Q$. It does not require that every part be non-empty. 
\end{frame}

\begin{frame}{Modularity Partitioning}
	There are modularity partitions which do not correspond to equilibria and there are equilibria which do not correspond to modularity partitions
	\begin{figure}
		\begin{tabular}{ccc}
			\begin{tikzpicture}
				\node (A) [draw = black, circle] at (1,0){$K_4$};
				\node (B) [circle,fill,inner sep=1.5pt]at (2,0){};
				\node (c) [circle,fill,inner sep=1.5pt]at (3,0){};
				\node (d) [circle,fill,inner sep=1.5pt]at (4,0.5){};
				\node (e) [circle,fill,inner sep=1.5pt]at (4,-0.5){};
				\node[fit=(A),dashed,draw, rectangle,rounded corners=10,inner sep=5pt] {};
				\node[fit=(B)(c)(d)(e),dashed, draw, rectangle,rounded corners=10,inner sep=5pt] {};
				
				\draw (1.3,0.34)--(2,0)--(1.3,-0.34);
				\draw(2,0)--(3,0)--(4,0.5)--(3,0)--(4,-0.5);
			\end{tikzpicture}
			&\hspace{0.5cm} &
			\begin{tikzpicture}
				\node (A) [draw = black, circle] at (1,0){$K_4$};
				\node (B) [circle,fill,inner sep=1.5pt]at (2,0){};
				\node (c) [circle,fill,inner sep=1.5pt]at (3,0){};
				\node (d) [circle,fill,inner sep=1.5pt]at (4,0.5){};
				\node (e) [circle,fill,inner sep=1.5pt]at (4,-0.5){};
				\node[fit=(A)(B),dashed,draw, rectangle,rounded corners=10,inner sep=5pt] {};
				\node[fit=(c)(d)(e),dashed, draw, rectangle,rounded corners=10,inner sep=5pt] {};
				
				\draw (1.3,0.34)--(2,0)--(1.3,-0.34);
				\draw(2,0)--(3,0)--(4,0.5)--(3,0)--(4,-0.5);
			\end{tikzpicture}\\
			$Q=0.2809$&&$Q=0.2603$\\
			Modularity Partition &&Not a Modularity Partition\\
			not an equilibrium&& equilibrium
		\end{tabular}
	\end{figure}
	
\end{frame}

\begin{frame}{Partitioning}
It is a relatively easy task to prove that no modularity partition can have a part with only 1 vertex. 
\begin{figure}
\begin{tikzpicture}
	\node (E1) at (0,0){};
	\node (E2) at (0,-1){};
	\node (EQU) at (3,-0.4){};
	\node (Q1) at (6,0){};
	\node (Q2) at (6,-1){};
	\node (U1) at (4,-2.5){};
	\node (U2) at (2,-2.5){};
	\node (U3) at (3,-3){};
	
	
	\node[fit=(E1)(EQU)(E2), draw, ellipse, fill = orange, opacity = 0.2 ,inner sep=2pt]{};
	\node[fit=(Q1)(EQU)(Q2), draw, ellipse,fill = blue, opacity = 0.2, inner sep=2pt]{};
	\node[fit=(U1)(EQU)(U2), draw, ellipse,inner sep=2pt]{};
	
	\node (Ename) at (0,0.75){Minimum Cut Partitions};
	\node(Qname) at (6,0.75){Modularity Partitions};
	\node(Uname) at (-0.2,-2.5){Equilibrium Partitions};
\end{tikzpicture}
\end{figure}
\end{frame}
\begin{frame}
	{A note on complexity class}
	Almost all partitioning problems are in complexity class NP. Finding a modularity paritition in particular can be shown to be NP complete\footnote{Brandes, U. Delling, D. Gaertler, M. G\"orke, R. Hoefer, M. Nikoloski Z. Wagner, D. ``On Modularity - NP-Completeness and Beyond"}. Because of the similarities between these partitions, it's unlikely that we can find an algorithm to solve the boundary value problem that runs deterministically in polynomial time unless the specification of boundary values reduces the problem more than expected. 
\end{frame}
\section{Continuous Variations and Applications}
\begin{frame}{Symptraic Evolution of Cooperative Behavior}
	\textbf{Sympatric speciation} is the process of speciation without geographic isolation. 
	
	These questions, modified to be continuous give us some tools to think about Sympatric evolution of cooperative behavior. 
\end{frame}

\begin{frame}{Sympatric Evolution of Cooperative Behavior}
	\begin{block}{Continuous Formulation 1, Continuous Player Space with Mixed Strategies}
		Let $\Omega$ be a domain and $\Delta_m$ be the $m$-simplex in which every mixed strategy lies. Let $\Phi:\Omega \rightarrow\Delta_m$ be a strategy profile. Where Payoffs are calculated as 
		\begin{equation}
			w(x|\Phi(x))=\sum_{i=1}^m \Phi_i(x)\int_\Omega K(x-y)\Phi_i(y)dy =:B(\Phi(x),\Phi(x)) 
		\end{equation} 
		Where $K$ is a familiarity kernel and $B$ is a bilinear form. 
	\end{block}
	Are there any non-constant, continuous $\Phi$ which are Nash Equilibria? 
\end{frame}

\begin{frame}{Sympatric Evolution of Cooperative Behavior}
	Are there any non-constant, continuous $\Phi$ which are Nash Equilibria?
	\begin{block}{Optimization Challenge}
		We are looking for continuous $\Phi$ such that for any $x$,
		\begin{equation}
			w(x|\Phi(x))\geq w(x|\tilde\Phi(x))\quad \forall \tilde \Phi(y) = \Phi (y) \text{ for } y\neq x
		\end{equation}. 
	\end{block}

	For the game posed above, perhaps there are no continuous non constant $\Phi$ which satisfy this condition. 
\end{frame}

\begin{frame}{Sympatric Evolution of Cooperative Behavior}
	\begin{block}{There are noconstant continuous solutions to (10)}
		Suppose that $\Phi^\star$ satisfies (10), we will show that it must be constant. 
	\end{block}
\end{frame}
	
\begin{frame}{Sympatric Evolution of Cooperative Behavior}
	This is not altogether surprising
	\begin{itemize}
		\item When strategies are discrete and the payoff matrix is the identity, then being around "nearby" strategies is not beneficial. 
		\item When strategies are discrete, the model is not entirely biologically relevant. 
	\end{itemize}
	Instead we ought to look at strategies which are comparable under a ``blurred" payoff matrix. 
\end{frame}

\begin{frame}{Discrete Player Space with comparable strategies}
	Consider this symmetric payoff matrix
	\begin{equation}
		\begin{bmatrix}
			1&\alpha&0\\
			\alpha&1&\alpha\\
			0&\alpha&1
		\end{bmatrix}
	\end{equation}
	\begin{figure} 
	\begin{tikzpicture}
		\node (A) at (0,0)[fill = blue, circle]{};
		\node (B) at (1,0)[fill = blue, circle]{};
		\node (C) at (2,0)[fill = orange, circle]{};
		\node (D) at (3,0)[fill = red, circle]{};
		\node (E) at (4,0)[fill = red, circle]{};
		
		\draw (A)--(B)--(C)--(D)--(E);
	\end{tikzpicture}
	\caption{This represents a Nash Equilibrium if and only if $\alpha\geq \frac{1}{2}$}
\end{figure}	
\end{frame}
	
\begin{frame}{Continuous Strategy and Continuous Player Space}
\end{frame}

\begin{frame}{Pure Coordination in the Continuous Setting}
\end{frame}

\begin{frame}{Metric Strategy Space}
\end{frame}


\end{document}

\section*{}
\begin{frame}{Conclusion}
	content...
\end{frame}