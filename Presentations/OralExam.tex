\documentclass{beamer}
\usetheme{berlin}
\definecolor{utorange}{HTML}{F77F00}
\definecolor{smokey}{HTML}{4C4D4F}
\definecolor{limestone}{HTML}{F0EDE4}
\definecolor{accent}{RGB}{0,018,147}
\definecolor{darkutorange}{HTML}{a95700}
\definecolor{lightutorange}{HTML}{ffe8d0}

\setbeamercolor*{title}{fg=smokey,bg=limestone}
\setbeamercolor*{author}{fg=smokey}
\setbeamercolor*{institute}{fg=smokey}
\setbeamercolor*{date}{fg=smokey}
\setbeamercolor*{frametitle}{fg=smokey, bg=limestone,}
\setbeamercolor*{structure}{fg=utorange}
\setbeamercolor{normal text}{fg=smokey,bg=white}
\setbeamercolor{alerted text}{fg=utorange}
\setbeamercolor{example text}{fg=smokey}
\setbeamercolor{palette primary}{fg=black, bg=utorange}
\setbeamercolor{palette secondary}{fg=white, bg=utorange}
\setbeamercolor{palette tertiary}{fg=white, bg=darkutorange}

\DeclareMathOperator*{\argmax}{\text{argmax}}
\DeclareMathOperator*{\argmin}{\text{argmin}}
\DeclareMathOperator{\uu}{\mathbf{u}}
\usepackage{tikz}
\usetikzlibrary{calc, shapes, fit}

\title{A Discrete Structured Coordination Game and its Applications in Theoretical Ecology}
\author{John McAlister}
\institute[Fefferman Lab]{Univeristy of Tennessee - Knoxville}
\begin{document}
	\begin{frame}[plain]
		\centering
		\maketitle
		\includegraphics[height=3cm]{images/LogoCenter.jpg}
	\end{frame}
\section{Introduction}
\begin{frame}{The Coordination Game}
	The Coordination game, at its most basic, is a two player game with the following payoff matrix
	\begin{center} 
		\begin{tabular}{c|cc}
			&A&B\\
			\hline 
			A&a,a&c,d\\
			B&d,c&b,b
		\end{tabular}
	\end{center}
	
	With the assumption that $a>d$ and $b>c$. It has two Pure Strategy Nash equilibria, $(A,A), (B,B)$, and a mixed strategy Nash equilibrium where both players play strategy $A$ with probability $p=\frac{b-d}{a+b-c-d}$.
\end{frame}
\begin{frame}{The Coordination Game}
	The game can also be considered among many players where pairs of players are selected to play uniformly randomly.
	\begin{itemize}
		\item Using myopic best response as a replicator dynamic, A group playing this game with \textit{high inertia} and $\varepsilon$\textit{-noise} will converge to the Risk Dominant Nash Equilibrium \footnote{Kandori, Michihiro, et al. “Learning, Mutation, and Long Run Equilibria in Games.”}
		
		\item Changing the way in which the pairs are selected can lead to different equilibrium selection \footnote{Robson, Arthur J., and Fernando Vega-Redondo. "Efficient equilibrium selection in evolutionary games with random matching."}
		
		\item Convergence time is very slow \footnote{Ellison, Glenn. "Basins of attraction, long-run stochastic stability, and the speed of step-by-step evolution."}
	\end{itemize}
\end{frame}
\begin{frame}{The Structured Coordination Game}
	If player selection is not uniform random across all players we call this a structured coordination game. 
	\begin{itemize}
		\item Complete Graphs (Identical to unstructured game)
		\item Circular Cities
		\item Square Lattice 
	\end{itemize}
	Each of these previous studies have taken the approach of using symmetry to reduce the state space, considering a transition matrix without noise and establishing a distance between states through mutation. \footnote{Ellison, Glenn. "Basins of attraction, long-run stochastic stability, and the speed of step-by-step evolution."}$^,$\footnote{Weidenholzer, Simon. "Coordination games and local interactions: a survey of the game theoretic literature."}
\end{frame}

\begin{frame}{The Structured Coordination Game}
	Consider this most general setting: For a connected graph $G(V,E)$ each vertex $v\in V$ plays a strategy $c$ from a set of pure strategies $C$. and the payoff for $v$ is given by \begin{equation}
		w(v,c|\uu)=|\{x\in \Gamma (v);\uu_x=c\}|
	\end{equation}
	where $\uu$ is the strategy profile and $\uu_x$ is the strategy that $x$ is using. 
	In this case the game is at its most simple where the Payoff matrix is simply $I_n$. Note, however, we are not limited to only 2 strategies as before. 
\end{frame}

\begin{frame}{The Structured Coordination Game as a Dynamical System}
	In keeping with the previous work, we use a myopic best response as our replicator dynamic. In this way we construct a sequence of strategy profiles $\uu(t)$ with 
	\begin{equation}
		\uu_v(t+1)\in \argmax_{c\in C}\{w(v,c|\uu(t))\}
	\end{equation}
It may be that $|\argmax|>1$ so we break ties in the following way ($\varepsilon$-inertia) 
\begin{itemize}
	\item if $\uu_v(t)\in \argmax_{c\in C}\{w(v,c|\uu(t))\}$ then $ \uu_v(t+1)=\uu_v(t)$.
	\item else, select from $\argmax_{c\in C}\{w(v,c|\uu(t))\}$ uniform randomly. 
	\end{itemize}
\end{frame}

\begin{frame}{Goals and Questions}
	It is clear to see that Equilibria in the dynamical system are Nash Equilibria of the game. There are many questions that arise from this system:
	\begin{itemize}
		\item Can we Characterize Equilibria for a general graph (or a particular topology) and determine stability conditions?
		
		\item Given a set of ``boundary conditions" can we find a strategic interpolation which is a Nash Equilibrium?
		
		\item What might a strategically continuous version of this game look like and what can it tell us about cooperative behaviors in ecology?
	\end{itemize}
\end{frame}
\section{Equilibria and Stability}
\begin{frame}{Equilibria}
	\begin{figure}
		\includegraphics[width=0.4\linewidth]{images/OralFig2v1}
		\includegraphics[width= 0.4\linewidth]{images/OralFig3v1}
		\caption{A nontrivial and trivial equilibrium in a 4-regular graph}
	\end{figure}
\end{frame}

\begin{frame}
	{Equilibra: Terminology}
	In order to discuss these ideas we need a definition
	\begin{block}{\textbf{Clique}}
		A subgraph of $G$ spanned by all of the vertices using a particular strategy. 
	\end{block}  
	and some notation
	\begin{block}{Notation}
		\begin{tabular}{rp{9cm}} 
			$Q(\mathbf{u})$& The set of all cliques in a strategy profile $\mathbf{u}$ \\
			$q^i$& A clique in $Q(\mathbf{u})$ in which vertices use strategy $i$\\
			$\partial q^i$& Vertices in $q_i$ which have neighbors in other cliques\\
		\end{tabular}
	\end{block}
\end{frame} 	
\begin{frame}{Preliminary Results}
	\begin{block}{The only equilibrium in $K_n$ is the trivial equilibrium}
		Suppose there is a equilibrium coloring $\mathbf{u^*}$ with $m\geq2$ cliques, $q^{c_1},...,q^{c_m}$. Because every pair of vertices shares an edge, $w(v,c_i|\uu)=|q_i|$ if $v\notin q^{c_i}$, and $w(v,c_i|\uu)=|q_i|-1$ if $v\in q^{c_i}$.  Consider a vertex $v_1\in q^{c_1}$. Because it is at equilibrium $$w(v_1,c_1|\uu^*)=\argmax_C\{w(v_1,c|\uu^*)\}=:a.$$
		Therefore $|q^{c_1}|=a+1$. Moreover $w(v_1,c_2|\uu^*)\leq a$ so $|q^{c_2}|\leq a$.
		Now consider a vertex  $v_2\in q^{c_2}$. $w(v_2,c_2|\uu^*)=|q^{c_2}|-1\leq a-1$ and $w(v_2,c_1|\uu^*)=a+1$. 
		Thus $\uu^*_2=c_2\notin \argmax\{w(v_2,c|\uu^*)\}$ so $\uu^*$ is not an equilibrium.  
	\end{block}
\end{frame}

\begin{frame}{Preliminary Results}
	\begin{block}{$K_{n,m}$ admits an equilibrium with $d$ cliques iff $d|n$ and $d|m$}
		$\impliedby$ It is easy to construct an equilibrium strategy profile with $d$ cliques
		
		$\implies$ I argue by contradiction. $K_{n,m}=E^n+E^m$  Without loss of generality suppose that $d$ is not a divisor of $m$. Suppose $\uu^*$ is an equilibrium strategy profile with $d>1$ cliques. $d$ does not divide $m$ so $\exists$ strategies $i$ and $j$ such that $|q^i\cap E^m|>|q^j\cap E^m|$. If $\uu^*$ is at equilibrium it must follow that $q^j\cap E^n=\emptyset$. It then follows $q^j\cap E^m =\emptyset$. If $q^j$ is empty then there are not $d$ cliques in $\uu^*$. This contradiction proves the result.   
	\end{block}
\end{frame}

\begin{frame}{More General Equilibria Results}
	In general the process of finding equilibria is equivalent to finding this kind of vertex partition.
	
	\begin{equation}
		P=\cup_{i=1}^m P^{c_i}\text{ where } x\in P^{c_i}\Rightarrow |\Gamma(x)\cap P^{c_i}|\geq |\Gamma(x)\cap P^{c_j}|\, \forall i,j 
	\end{equation}
	There are other similar kinds of graph partitions
	\begin{itemize}
		\item Ratio Partitions
		\item Maximum Modularity Partitions
		\item balanced (k,v) Partitions
	\end{itemize}
\end{frame}

\begin{frame}{Partitioning Observations}
	There are modularity partitions which do not correspond to equilibria and there are equilibria which do not correspond to modularity partitions
\begin{figure}
	\begin{tabular}{ccc}
		\begin{tikzpicture}
			\node (A) [draw = black, circle] at (1,0){$K_4$};
			\node (B) [circle,fill,inner sep=1.5pt]at (2,0){};
			\node (c) [circle,fill,inner sep=1.5pt]at (3,0){};
			\node (d) [circle,fill,inner sep=1.5pt]at (4,0.5){};
			\node (e) [circle,fill,inner sep=1.5pt]at (4,-0.5){};
			 \node[fit=(A),dashed,draw, rectangle,rounded corners=10,inner sep=5pt] {};
			 \node[fit=(B)(c)(d)(e),dashed, draw, rectangle,rounded corners=10,inner sep=5pt] {};
			
			\draw (1.3,0.34)--(2,0)--(1.3,-0.34);
			\draw(2,0)--(3,0)--(4,0.5)--(3,0)--(4,-0.5);
		\end{tikzpicture}
		&\hspace{0.5cm} &
		\begin{tikzpicture}
			\node (A) [draw = black, circle] at (1,0){$K_4$};
			\node (B) [circle,fill,inner sep=1.5pt]at (2,0){};
			\node (c) [circle,fill,inner sep=1.5pt]at (3,0){};
			\node (d) [circle,fill,inner sep=1.5pt]at (4,0.5){};
			\node (e) [circle,fill,inner sep=1.5pt]at (4,-0.5){};
			\node[fit=(A)(B),dashed,draw, rectangle,rounded corners=10,inner sep=5pt] {};
			\node[fit=(c)(d)(e),dashed, draw, rectangle,rounded corners=10,inner sep=5pt] {};
			
			\draw (1.3,0.34)--(2,0)--(1.3,-0.34);
			\draw(2,0)--(3,0)--(4,0.5)--(3,0)--(4,-0.5);
		\end{tikzpicture}\\
		$Q=0.2809$&&$Q=0.2603$\\
		Modularity Partition &&Not a Modularity Partition\\
		not an equilibrium&& equilibrium
	\end{tabular}
\end{figure}
	
	
	
%	\begin{block}{If $G$ has an equilibrium with $d$ cliques, then a Maximal Modularity Partition of $d$ parts corresponds to an equilibrium.}
%		Make a correspondence between a vertex partition $\mathcal{P}=\cup_{i=1}^dP_i$ and strategy profile $\uu$ by assigning strategy $c_i$ to all vertices in $P_i$ for all $i$.
%		Suppose the above is not true and a Maximal Modularity Partition $\mathcal{P}$ corresponds to $\uu^*$ which is not an equilibrium. Therefore in the corresponding $\uu^*$ there is at least one vertex $v\in P_i$ with $w(v,c_i,\uu)<w(v,c_j,\uu)$ for some $c_j\neq c_i$. Let $\hat\uu$ be the strategy profile wherein $\hat\uu_i=\uu_i$ for $i\neq v$ and $\hat\uu_v = c_j$. 
%		A tedious computations shows that the modularity of the partition $\hat{\mathcal{P}}$ corresponding to the strategy profile $\hat\uu$ is higher than that of $\mathcal{P}$.  
%	\end{block}
\end{frame}

\begin{frame}{Stability}
	When viewed as a dynamical system, stability can be considered in multiple ways 
	\begin{itemize}
		\item Local Stability - A vertex is stable to a single perturbation
		\item Global Stability - Every vertex is stable to a single perturbation
		\item Convergent Stability - Every ``nearby" strategy profile evolves into this strategy profile
		\item Structural Stability - A strategy profile is stable to perturbations of the graph structure. 
	\end{itemize}
\end{frame}

\begin{frame}{Cycles}
	When viewed as a dynamical system we also see the (common) emergence of cycles. We conjecture that $n$-cycles with $n>2$ are impossible. 
	\begin{figure}
	\includegraphics[width = 0.8\linewidth]{images/Figure6}
	\end{figure} 
\end{frame}

\begin{frame}{Goals and Further Questions}
	We hope to 
	\begin{itemize}
		\item Prove the $n$-cycle conjecture,
		\item Find analytically tractable stability criteria
		\item Use energy-like estimates to describe basins of stability for equilibria
		\item Use techniques from extremal graph theory to describe possible equilibria by the structure of the graph by which they are admitted
		\item Summarize with some computational results relating edge density or connectance to clique number 
	\end{itemize}
\end{frame}


\section{Boundary Value Problem}
\begin{frame}{Boundary Value Problem}
	A vital application of this idea is a boundary value conception of the problem.
	\begin{block}{Boundary Value Problem}
		Suppose $B\subset V$ is a subset of vertices which are assigned strategies by $f:B\rightarrow C$. Is there a strategy profile such that $\uu_v=f(v)$ for all $v\in B$ and $\uu$ is an equilibrium strategy profile?
	\end{block}
\end{frame}
\begin{frame}{Intuition}
	Suppose that in  a signaling network $G$
	\begin{itemize}
		\item Each vertex uses exactly one language to send a receive signals. 
		\item A subset of vertices are assigned languages
		\item translation of the signal is costly relative to transmission. 
	\end{itemize}

	How can we assign a ``language" to each vertex so that each vertex is minimizing their own ``translation" burden?
\end{frame}

\begin{frame}{Steps Forward}
	This is rather speculative but the direction forward may look like this:
	\begin{itemize}
		\item Trying to prove existence of such an equilibrium interpolation for an admissible set of boundary values. 
		\item (Almost Equivalently) finding a class of boundary values for which an interpolation can be made
		\item Seeking out an algorithm to build an equilibrium interpolation through graph reductions.  
	\end{itemize}
\end{frame}
\begin{frame}{Steps Forward}
	As before this becomes a partitioning problem but with an added components
	\begin{block}{Boundary Value Problem }
		For a graph $G(V,E)$, $B\subset V$ and $F:B\rightarrow C$, find a partition $\mathcal{P}=\cup_{i=1}^mP^{c_i}$ such that 
		\begin{equation}
			\begin{split}
				i)\quad & |\Gamma(x)\cap P^{\uu_x}|\geq |\Gamma(x)\cap P^{c}|\,\forall x\in V, c\in C \\
				ii)\quad &  x\in P^{f(x)}\,\forall x\in B 
			\end{split}
		\end{equation}
	\end{block}
	The brute force algorithm has time complexity $\mathcal{O}(n^n)$, so we can only get better from there. 
\end{frame}
\begin{frame}
	{A note on complexity class}
	Almost all partitioning problems are in complexity class NP, some have been shown to be NP complete. It's unlikely, unless P=NP that an algorithm so solve the boundary value problem could run deterministically polynomial time.
\end{frame}
\section{Continuous Variations and applications}
\begin{frame}{Symptraic Evolution of cooperative behavior}
	content...
\end{frame}

\end{document}
