\documentclass{beamer}
\usetheme{berlin}
\definecolor{utorange}{HTML}{F77F00}
\definecolor{smokey}{HTML}{4C4D4F}
\definecolor{limestone}{HTML}{F0EDE4}
\definecolor{accent}{RGB}{0,018,147}
\definecolor{darkutorange}{HTML}{a95700}
\definecolor{lightutorange}{HTML}{ffe8d0}

\setbeamercolor*{title}{fg=smokey,bg=limestone}
\setbeamercolor*{author}{fg=smokey}
\setbeamercolor*{institute}{fg=smokey}
\setbeamercolor*{date}{fg=smokey}
\setbeamercolor*{frametitle}{fg=smokey, bg=limestone,}
\setbeamercolor*{structure}{fg=utorange}
\setbeamercolor{normal text}{fg=smokey,bg=white}
\setbeamercolor{alerted text}{fg=utorange}
\setbeamercolor{example text}{fg=smokey}
\setbeamercolor{palette primary}{fg=black, bg=utorange}
\setbeamercolor{palette secondary}{fg=white, bg=utorange}
\setbeamercolor{palette tertiary}{fg=white, bg=darkutorange}

\DeclareMathOperator*{\argmax}{\text{argmax}}
\DeclareMathOperator*{\argmin}{\text{argmin}}
\DeclareMathOperator{\uu}{\mathbf{u}}
\usepackage{tikz}
\usepackage{array}
\usetikzlibrary{calc, shapes, fit}


\title{A Structured Coordination Game With Neutral Options}
\author{John McAlister}
\institute[Fefferman Lab]{University of Tennessee - Knoxville}
\date{March 16, 2024}
\begin{document}
\begin{frame}[plain]
    \centering
    \maketitle
    \includegraphics[height=3cm]{images/LogoCenter.jpg}
\end{frame}
\section{Introduction}
\begin{frame}{The Coordination Game}
		The coordination game with neutral options, is an incredibly simple game:
	\begin{center} 
		\begin{tabular}{c|cc}
			&A&B\\
			\hline 
			A&1,1&0,0\\
			B&0,0&1,1
		\end{tabular}
	\end{center}
	
	There are two pure strategy Nash equilibria $(A,A), (B,B)$, and a mixed strategy Nash equilibrium where both players play strategy $A$ with probability $p=\frac{1}{2}$.
\end{frame}

\begin{frame}{The Structured Coordination Game}
	When there are many players playing this game and interacting heterogeneously we can imagine the interaction taking place on a graph
	\begin{figure}
		\includegraphics[width=0.6\linewidth]{images/oralfig3v4}
	\end{figure}
\end{frame}

\begin{frame}{Partitioning}
	There is a very natural correspondence between strategy profiles and graph partitions 
	\begin{block}{Strategic Partitions}
		For a strategy profile $\uu:V\rightarrow C$ the corresponding partition is $Q:=\{q^c\}_{c\in C}$ where $q^c=\{v\in V|\uu(v)=c\}$
	\end{block}  
	\begin{block}{Equilibrium Partition}
		We define a partition to be an equilibrium partition if it corresponds to a strategy profile which is a Nash equilibrium. 
	\end{block}
\end{frame}
\begin{frame}{Some small partitioning results}
	\begin{itemize}
		\item For every graph, the trivial partition,$Q=\{V\}$ is an equilibrium partition.
		\item If $Q=\{q^c\}_{c\in C}$ is an equilibrium partition of a connected graph on at least 2 vertices, then $|q^c|\neq 1$ for all $c\in C$. 
		\item If $Q=\{q^c\}_{c\in C}$ is an equilibrium partition and the subgraph spanned by $q^i$ is disconnected, then, if $q^i_1,...q^i_n$ are the connected components of $q^i$, $\tilde{Q}=\{q^c\}_{c\neq i}\cup \{q^i_j\}_{j=1}^n$ is also an equilibrium partition. 
	\end{itemize}
\end{frame}

\section{Catalogue}
\begin{frame}{Methods}
	Seeking to build a catalogue of equilibrium partitions for small graphs, we use a brute force method
	\begin{block}{Checking if a partition is an equilibrium partition is $\mathcal{P}$ hard.}
		We need only calculate the best response of each vertex in graph. If every vertex is playing its best response, then it is a Nash equilibrium and thus an equilibrium partition. This can be done with in at most $n^2$ operations. 
	\end{block} 
\end{frame}
\begin{frame}{Checking for Isomorphisms}
	
	\begin{figure}
		\centering
		\scalebox{0.8}{
		\begin{tabular}{|m{2cm}|m{5.5cm}|c|}
			\hline
			Type& Partition Diagram& Vector View \\
			\hline
			\hline
			\raggedright
			Isomorphic by way of relabeling & \begin{tikzpicture}
				\node(a)[circle, fill, inner sep =1.5pt] at (0,0){};
				\node(b)[circle, fill, inner sep = 1.5pt] at(0.5,-0.5){};
				\node(c)[circle, fill, inner sep = 1.5pt] at(1,0){};
				\node(d)[circle, fill, inner sep = 1.5pt] at(1.5,-0.5){};
				\node(e)[circle, fill, inner sep = 1.5pt] at(2,0){};
				
				\draw (a)--(b)--(c)--(d)--(e);
				
				\node[fit=(a)(b)(c),dashed, draw, rectangle,rounded corners=10,inner sep=5pt] {};
				\node[fit=(d)(e),dashed, draw, rectangle,rounded corners=10,inner sep=5pt] {};
				
				\node(label11) at (0.5,0.5){$p^1$};
				\node(label12) at (1.75,0.5){$p^2$};
				
				\node(equal)at (2.5,-0.25){$\simeq$};
				
				\node(f)[circle, fill, inner sep =1.5pt] at (3,0){};
				\node(g)[circle, fill, inner sep = 1.5pt] at(3.5,-0.5){};
				\node(h)[circle, fill, inner sep = 1.5pt] at(4,0){};
				\node(i)[circle, fill, inner sep = 1.5pt] at(4.5,-0.5){};
				\node(j)[circle, fill, inner sep = 1.5pt] at(5,0){};
				
				\draw (f)--(g)--(h)--(i)--(j);
				
				\node[fit=(f)(g)(h),dashed, draw, rectangle,rounded corners=10,inner sep=5pt] {};
				\node[fit=(i)(j),dashed, draw, rectangle,rounded corners=10,inner sep=5pt] {};
				
				\node(label21) at (3.5,0.5){$p^2$};
				\node(label22) at (4.75,0.5){$p^1$};
				
			\end{tikzpicture} 
			& $[1,1,1,2,2]\simeq[2,2,2,1,1]$\\
			\hline \raggedright
			Isomorphic by way of symmetry & \begin{tikzpicture}
				\node(a)[circle, fill, inner sep =1.5pt] at (0,0){};
				\node(b)[circle, fill, inner sep = 1.5pt] at(0.5,-0.5){};
				\node(c)[circle, fill, inner sep = 1.5pt] at(1,0){};
				\node(d)[circle, fill, inner sep = 1.5pt] at(1.5,-0.5){};
				\node(e)[circle, fill, inner sep = 1.5pt] at(2,0){};
				
				\draw (a)--(b)--(c)--(d)--(e);
				
				\node[fit=(a)(b)(c),dashed, draw, rectangle,rounded corners=10,inner sep=5pt] {};
				\node[fit=(d)(e),dashed, draw, rectangle,rounded corners=10,inner sep=5pt] {};
				
				\node(label11) at (0.5,0.5){$p^1$};
				\node(label12) at (1.75,0.5){$p^2$};
				
				\node(equal)at (2.5,-0.25){$\simeq$};
				
				\node(f)[circle, fill, inner sep =1.5pt] at (3,0){};
				\node(g)[circle, fill, inner sep = 1.5pt] at(3.5,-0.5){};
				\node(h)[circle, fill, inner sep = 1.5pt] at(4,0){};
				\node(i)[circle, fill, inner sep = 1.5pt] at(4.5,-0.5){};
				\node(j)[circle, fill, inner sep = 1.5pt] at(5,0){};
				
				\draw (f)--(g)--(h)--(i)--(j);
				
				\node[fit=(f)(g),dashed, draw, rectangle,rounded corners=10,inner sep=5pt] {};
				\node[fit=(h)(i)(j),dashed, draw, rectangle,rounded corners=10,inner sep=5pt] {};
				
				\node(label21) at (3.25,0.5){$p^2$};
				\node(label22) at (4.5,0.5){$p^1$};
				
			\end{tikzpicture} 
			& $[1,1,1,2,2]\simeq[2,2,1,1,1]$\\
			\hline\raggedright
			Isomorphic by way of symmetry and relabeling & \begin{tikzpicture}
				\node(a)[circle, fill, inner sep =1.5pt] at (0,0){};
				\node(b)[circle, fill, inner sep = 1.5pt] at(0.5,-0.5){};
				\node(c)[circle, fill, inner sep = 1.5pt] at(1,0){};
				\node(d)[circle, fill, inner sep = 1.5pt] at(1.5,-0.5){};
				\node(e)[circle, fill, inner sep = 1.5pt] at(2,0){};
				
				\draw (a)--(b)--(c)--(d)--(e);
				
				\node[fit=(a)(b)(c),dashed, draw, rectangle,rounded corners=10,inner sep=5pt] {};
				\node[fit=(d)(e),dashed, draw, rectangle,rounded corners=10,inner sep=5pt] {};
				
				\node(label11) at (0.5,0.5){$p^1$};
				\node(label12) at (1.75,0.5){$p^2$};
				
				\node(equal)at (2.5,-0.25){$\simeq$};
				
				\node(f)[circle, fill, inner sep =1.5pt] at (3,0){};
				\node(g)[circle, fill, inner sep = 1.5pt] at(3.5,-0.5){};
				\node(h)[circle, fill, inner sep = 1.5pt] at(4,0){};
				\node(i)[circle, fill, inner sep = 1.5pt] at(4.5,-0.5){};
				\node(j)[circle, fill, inner sep = 1.5pt] at(5,0){};
				
				\draw (f)--(g)--(h)--(i)--(j);
				
				\node[fit=(f)(g),dashed, draw, rectangle,rounded corners=10,inner sep=5pt] {};
				\node[fit=(h)(i)(j),dashed, draw, rectangle,rounded corners=10,inner sep=5pt] {};
				
				\node(label21) at (3.25,0.5){$p^1$};
				\node(label22) at (4.5,0.5){$p^2$};
				
			\end{tikzpicture} 
			& $[1,1,1,2,2]\simeq[1,1,2,2,2]$\\
			\hline
		\end{tabular}
		}	
		\caption{An example of the three different ways two labeled partitions of labeled graphs can be isomorphic to one another and the "vector view" which is how the computer stores the partition information.}
		\label{isofigure}
	\end{figure}
	
\end{frame}
\begin{frame}{Results}
	\begin{table}[]
		\centering
		\begin{tabular}{c|cccccccccc}
			&\multicolumn{10}{|c}{number of graphs with $n$ partitions}  \\
			\hline 
			Graph size&1&2&3&4&5&6&7&8&9&10\\
			\hline 
			1&1&0&0&0&0&0&0&0&0&0\\
			2&1&0&0&0&0&0&0&0&0&0\\
			3&2&0&0&0&0&0&0&0&0&0\\
			4&4&2&0&0&0&0&0&0&0&0\\
			5&13&8&0&0&0&0&0&0&0&0\\
			6&48&43&13&6&2&0&0&0&0&0\\
			7&319&297&128&56&25&15&8&3&1&1\\
			\hline
			Total&399&350&141&62&27&15&8&3&1&1
		\end{tabular}
		\caption{A table showing the number of connected graphs which admit $n$ different equilibrium partitions for $n$ from 1 to 10 up to isomorphism. Among graphs of size less or equal to seven, there are no graphs which admit more than 10 different partitions.}
		\label{tab:CountPartitionNumber}
	\end{table}
\end{frame}

\begin{frame}{Results}
	\begin{table}[h!]
		\centering
		\begin{tabular}{c|cccccc}
			& \multicolumn{6}{|c}{number of partitions with $n$ parts} \\
			\hline 
			Graph Size&\multicolumn{2}{|c}{1}&\multicolumn{2}{c}{2}&\multicolumn{2}{c}{3}\\
			\hline 
			1&1&100\%&0&0\%&0&0\%\\
			2&1&100\%&0&0\%&0&0\%\\
			3&2&100\%&0&0\%&0&0\%\\
			4&6&75\%&2&25\%&0&0\%\\
			5&21&72\%&8&27\%&0&0\%\\
			6&112&54\%&79&38\%&16&8\%\\
			7&853&46\%&808&44\%&174&9\%\\
			\hline
			Total& 996&48\%&897&43\%&190&9\%
		\end{tabular}
		\caption{A table showing the number of distinct partitions with $n$ parts for each graph size from 1 to 7 vertices. }
		\label{NumberClusters}
	\end{table}
\end{frame}
\begin{frame}{Results}
	\begin{figure}
		\centering
		\scalebox{0.8}{
		\begin{tabular}{ccccc}
			\begin{tikzpicture}
				\node(a)[circle, fill, inner sep =1.5pt] at (0,0){};
				\node(b)[circle, fill, inner sep = 1.5pt] at(0.66,0.2){};
				\node(c)[circle, fill, inner sep = 1.5pt] at(1.33,0.2){};
				\node(d)[circle, fill, inner sep = 1.5pt] at(1.9,-0.5){};
				\node(e)[circle, fill, inner sep = 1.5pt] at(1.33,-1.2){};
				\node(f)[circle, fill, inner sep = 1.5pt] at(0.66,-1.2){};
				\node(g)[circle, fill, inner sep = 1.5pt] at(0,-1){};
				
				
				
				\draw (a)--(b)--(c)--(d)--(e)--(f)--(g)--(a);
				\draw (b)--(f);
				
				\node[fit=(a)(b)(c)(d)(e)(f)(g),dashed, draw, rectangle,rounded corners=10,inner sep=5pt] {};
			\end{tikzpicture}& 
			\begin{tikzpicture}
				\node(a)[circle, fill, inner sep =1.5pt] at (0,0){};
				\node(b)[circle, fill, inner sep = 1.5pt] at(0.66,0.2){};
				\node(c)[circle, fill, inner sep = 1.5pt] at(1.33,0.2){};
				\node(d)[circle, fill, inner sep = 1.5pt] at(1.9,-0.5){};
				\node(e)[circle, fill, inner sep = 1.5pt] at(1.33,-1.2){};
				\node(f)[circle, fill, inner sep = 1.5pt] at(0.66,-1.2){};
				\node(g)[circle, fill, inner sep = 1.5pt] at(0,-1){};
				
				
				
				\draw (a)--(b)--(c)--(d)--(e)--(f)--(g)--(a);
				\draw (b)--(f);
				
				\node[fit=(a)(b)(c),dashed, draw, rectangle,rounded corners=10,inner sep=5pt] {};
				\node[fit=(d)(e)(f)(g),dashed, draw, rectangle,rounded corners=10,inner sep=5pt] {};
			\end{tikzpicture}& 
			\begin{tikzpicture}
				\node(a)[circle, fill, inner sep =1.5pt] at (0,0){};
				\node(b)[circle, fill, inner sep = 1.5pt] at(0.66,0.2){};
				\node(c)[circle, fill, inner sep = 1.5pt] at(1.33,0.2){};
				\node(d)[circle, fill, inner sep = 1.5pt] at(1.9,-0.5){};
				\node(e)[circle, fill, inner sep = 1.5pt] at(1.33,-1.2){};
				\node(f)[circle, fill, inner sep = 1.5pt] at(0.66,-1.2){};
				\node(g)[circle, fill, inner sep = 1.5pt] at(0,-1){};
				
				
				
				\draw (a)--(b)--(c)--(d)--(e)--(f)--(g)--(a);
				\draw (b)--(f);
				
				\node[fit=(a)(g),dashed, draw, rectangle,rounded corners=10,inner sep=5pt] {};
				\node[fit=(b)(c)(d)(e)(f),dashed, draw, rectangle,rounded corners=10,inner sep=5pt] {};
			\end{tikzpicture}&
			\begin{tikzpicture}
				\node(a)[circle, fill, inner sep =1.5pt] at (0,0){};
				\node(b)[circle, fill, inner sep = 1.5pt] at(0.66,0.2){};
				\node(c)[circle, fill, inner sep = 1.5pt] at(1.33,0.2){};
				\node(d)[circle, fill, inner sep = 1.5pt] at(1.9,-0.5){};
				\node(e)[circle, fill, inner sep = 1.5pt] at(1.33,-1.2){};
				\node(f)[circle, fill, inner sep = 1.5pt] at(0.66,-1.2){};
				\node(g)[circle, fill, inner sep = 1.5pt] at(0,-1){};
				
				
				
				\draw (a)--(b)--(c)--(d)--(e)--(f)--(g)--(a);
				\draw (b)--(f);
				
				\node[fit=(a)(g)(b)(f),dashed, draw, rectangle,rounded corners=10,inner sep=5pt] {};
				\node[fit=(c)(d)(e),dashed, draw, rectangle,rounded corners=10,inner sep=5pt] {};
			\end{tikzpicture}&    \begin{tikzpicture}
				\node(a)[circle, fill, inner sep =1.5pt] at (0,0){};
				\node(b)[circle, fill, inner sep = 1.5pt] at(0.66,0.2){};
				\node(c)[circle, fill, inner sep = 1.5pt] at(1.2,0.2){};
				\node(d)[circle, fill, inner sep = 1.5pt] at(1.9,-0.5){};
				\node(e)[circle, fill, inner sep = 1.5pt] at(1.66,-1.2){};
				\node(f)[circle, fill, inner sep = 1.5pt] at(0.66,-1.2){};
				\node(g)[circle, fill, inner sep = 1.5pt] at(0,-1){};
				
				
				
				\draw (a)--(b)--(c)--(d)--(e)--(f)--(g)--(a);
				\draw (b)--(f);
				
				\node[fit=(a)(g)(b)(f)(c),dashed, draw, rectangle,rounded corners=10,inner sep=5pt] {};
				\node[fit=(d)(e),dashed, draw, rectangle,rounded corners=10,inner sep=5pt] {};
			\end{tikzpicture}\\
			
			\begin{tikzpicture}
				\node(a)[circle, fill, inner sep =1.5pt] at (0,0){};
				\node(b)[circle, fill, inner sep = 1.5pt] at(0.66,0.2){};
				\node(c)[circle, fill, inner sep = 1.5pt] at(1.33,0.2){};
				\node(d)[circle, fill, inner sep = 1.5pt] at(1.9,-0.5){};
				\node(e)[circle, fill, inner sep = 1.5pt] at(1.33,-1.2){};
				\node(f)[circle, fill, inner sep = 1.5pt] at(0.66,-1.2){};
				\node(g)[circle, fill, inner sep = 1.5pt] at(0,-1){};
				
				
				
				\draw (a)--(b)--(c)--(d)--(e)--(f)--(g)--(a);
				\draw (b)--(f);
				
				\node[fit=(a)(g),dashed, draw, rectangle,rounded corners=10,inner sep=5pt] {};
				\node[fit=(b)(c)(d),dashed, draw, rectangle,rounded corners=10,inner sep=5pt] {};
				\node[fit=(f)(e),dashed, draw, rectangle,rounded corners=10,inner sep=5pt] {};
			\end{tikzpicture}&
			\begin{tikzpicture}
				\node(a)[circle, fill, inner sep =1.5pt] at (0,0){};
				\node(b)[circle, fill, inner sep = 1.5pt] at(0.66,0.2){};
				\node(c)[circle, fill, inner sep = 1.5pt] at(1.33,0.2){};
				\node(d)[circle, fill, inner sep = 1.5pt] at(1.9,-0.5){};
				\node(e)[circle, fill, inner sep = 1.5pt] at(1.33,-1.2){};
				\node(f)[circle, fill, inner sep = 1.5pt] at(0.66,-1.2){};
				\node(g)[circle, fill, inner sep = 1.5pt] at(0,-1){};
				
				
				
				\draw (a)--(b)--(c)--(d)--(e)--(f)--(g)--(a);
				\draw (b)--(f);
				
				\node[fit=(a)(b),dashed, draw, rectangle,rounded corners=10,inner sep=5pt] {};
				\node[fit=(g)(f),dashed, draw, rectangle,rounded corners=10,inner sep=5pt] {};
				\node[fit=(c)(d)(e),dashed, draw, rectangle,rounded corners=10,inner sep=5pt] {};
			\end{tikzpicture}&
			\begin{tikzpicture}
				\node(a)[circle, fill, inner sep =1.5pt] at (0,0){};
				\node(b)[circle, fill, inner sep = 1.5pt] at(0.66,0.2){};
				\node(c)[circle, fill, inner sep = 1.5pt] at(1.33,0.2){};
				\node(d)[circle, fill, inner sep = 1.5pt] at(1.9,-0.5){};
				\node(e)[circle, fill, inner sep = 1.5pt] at(1.33,-1.2){};
				\node(f)[circle, fill, inner sep = 1.5pt] at(0.66,-1.2){};
				\node(g)[circle, fill, inner sep = 1.5pt] at(0,-1){};
				
				
				
				\draw (a)--(b)--(c)--(d)--(e)--(f)--(g)--(a);
				\draw (b)--(f);
				
				\node[fit=(a)(b),dashed, draw, rectangle,rounded corners=10,inner sep=5pt] {};
				\node[fit=(g)(f)(e),dashed, draw, rectangle,rounded corners=10,inner sep=5pt] {};
				\node[fit=(c)(d),dashed, draw, rectangle,rounded corners=10,inner sep=5pt] {};
			\end{tikzpicture}&
			\begin{tikzpicture}
				\node(a)[circle, fill, inner sep =1.5pt] at (0,0){};
				\node(b)[circle, fill, inner sep = 1.5pt] at(0.66,0.2){};
				\node(c)[circle, fill, inner sep = 1.5pt] at(1.33,0.2){};
				\node(d)[circle, fill, inner sep = 1.5pt] at(1.9,-0.5){};
				\node(e)[circle, fill, inner sep = 1.5pt] at(1.33,-1.2){};
				\node(f)[circle, fill, inner sep = 1.5pt] at(0.66,-1.2){};
				\node(g)[circle, fill, inner sep = 1.5pt] at(0,-1){};
				
				
				
				\draw (a)--(b)--(c)--(d)--(e)--(f)--(g)--(a);
				\draw (b)--(f);
				
				\node[fit=(a)(g),dashed, draw, rectangle,rounded corners=10,inner sep=5pt] {};
				\node[fit=(b)(f),dashed, draw, rectangle,rounded corners=10,inner sep=5pt] {};
				\node[fit=(c)(d)(e),dashed, draw, rectangle,rounded corners=10,inner sep=5pt] {};
			\end{tikzpicture}&
			\begin{tikzpicture}
				\node(a)[circle, fill, inner sep =1.5pt] at (0,0){};
				\node(b)[circle, fill, inner sep = 1.5pt] at(0.66,0.2){};
				\node(c)[circle, fill, inner sep = 1.5pt] at(1.2,0.2){};
				\node(d)[circle, fill, inner sep = 1.5pt] at(1.9,-0.5){};
				\node(e)[circle, fill, inner sep = 1.5pt] at(1.66,-1.2){};
				\node(f)[circle, fill, inner sep = 1.5pt] at(0.66,-1.2){};
				\node(g)[circle, fill, inner sep = 1.5pt] at(0,-1){};
				
				
				
				\draw (a)--(b)--(c)--(d)--(e)--(f)--(g)--(a);
				\draw (b)--(f);
				
				\node[fit=(a)(g),dashed, draw, rectangle,rounded corners=10,inner sep=5pt] {};
				\node[fit=(b)(f)(c),dashed, draw, rectangle,rounded corners=10,inner sep=5pt] {};
				\node[fit=(d)(e),dashed, draw, rectangle,rounded corners=10,inner sep=5pt] {};
			\end{tikzpicture}     
		\end{tabular}
	}
		\caption{The ten distinct equilibrium partitions of the graph $X_{38}$ which is graph \#445 in the graph atlas. This is the only graph among those catalogued which admits ten distinct equilibrium partitions.}
		\label{graph445}
	\end{figure}
\end{frame}
\section{Basins of Stability}
\begin{frame}{Simulation Methods}
	On larger graphs, It becomes impractical to check every partition so we take a different approach to finding equilibria
	\begin{block}{Initial Value Problem}
		Given a strategy profile $\uu(t)$ compute the next strategy profile by computing best responses.
		\begin{equation*}
			\uu(t+1)=\big[\argmax_{c\in C}|\{x\in\Gamma(v); \uu_x(t)=c\} |]_{v\in V}
		\end{equation*}
		This gives a (non-unique) solution to the initial value problem resulting in an equilibrium or $n$-cycle\footnote{or the sequence does not converge to a limit}.
	\end{block}
\end{frame}
\begin{frame}{Simulation Methods}
	Each Nash equilibrium (and corresponding equilibrium partition) therefore has a (possibly overlapping) basin of stability. One crucial question is to understand the size of the basin of stability corresponding to the trivial equilibrium partition. 
	\begin{block}{Simulation Design}
		For each Erd\'os-Renyi random graph, solve the IVP for 500 random initial conditions and record how many of them result in a trivial partition. 
	\end{block}
\end{frame}
\begin{frame}{Results}
	con
\end{frame}
\section{Broader Simulation}
\begin{frame}{Simulation Methods}
	
\end{frame}
\begin{frame}{Results}
	content...
\end{frame}
\begin{frame}{Observations and Conjectures}
	\begin{itemize}
		\item The computation time for solving the IVP is linear in graph size
		\item $n$-cycles occur with probability zero for $n>2$. 
	\end{itemize}
\end{frame}
\end{document}
