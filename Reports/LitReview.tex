\documentclass[]{article}
\usepackage{amssymb}
\usepackage{amsmath}
\usepackage{amsthm}
\usepackage{graphicx}


\newtheorem{theorem}{Theorem}
\newtheorem{proposition}[theorem]{Proposition}
\newtheorem{conjecture}[theorem]{Conjecture}
\newtheorem{corrollary}[theorem]{Corollary}
\newtheorem{lemma}[theorem]{Lemma}

%opening
\title{A Review of Structured Coordination Games and Their Applications to Theoretical Ecology}
\author{John McAlister}

\begin{document}

\maketitle

\begin{abstract}

\end{abstract}

\section{Simple Coordination Games}
	%introduce what the game is the analysis of its equilibria and the analysis of the game in a repeated setting. 
	At is most basic, A coordination game is an interaction between two players who each have two choices $A$ and $B$. If both players pick the same choice they will get a positive payoff and if they do not they will get zero payoff \cite{Russell1999, Weibull1995}
	\begin{table}[h]\centering
		\begin{tabular}{c|cc}
			& A&B\\
			\hline
			A&(1,1)&(0,0)\\
			B&(0,0)&(1,1)
		\end{tabular}
	\caption{The payoff matrix for the most simple coordination game}
	\end{table}
	
	In this very simple setting there are clearly two pure strategy Nash Equilibria (NE) when the two players use the same strategy. We can show that there is also a mixed strategy equilibrium when both players play each strategy with even probability. 
	
	\begin{proposition}[Mixed Strategy Equilibrium]
		The simple coordination game in Table 1 has a mixed strategy equilibrium in which both players play each strategy with probability 1/2.
	\end{proposition}
	\begin{proof}
		Suppose Player 1 plays $A$ with probability $P_1(A)=p_1$ and player 2 plays $A$ with probability $P_2(A)=p_2$. At a mixed strategy NE Player 2's expected payoffs (E) should be equal so we require $E[A]=p_1=E[B]=(1-p_1)$. Thus, there will only be a mixed strategy equilibrium if and only if $p_1=1/2$. Likewise Player 2's expected payoffs must be equal so we require $E[A]=p_2=E[B]=1-p_2$. Therefore $p_2=1/2$. Thus we conclude there is a mixed strategy NE when both players play $A$ with probability 1/2.  
	\end{proof}
	
	We can make this game more complicated by making the payoffs more general. For instance consider the general coordination game
	\begin{table}[h]\centering
		\begin{tabular}{c|cc}
			& A&B\\
			\hline
			A&(a,a)&(d,c)\\
			B&(c,d)&(b,b)
		\end{tabular}
		\caption{The payoff matrix for the coordination game. Player 1's strategies are listed on the left side of the matrix and player 2's are listed on top. Each payoff is listed as (Player 1's, Player 2's)}
	\end{table}
	
	Here we get a pure strategy NE (A,A) whenever $d\leq a$ and a pure strategy NE (B,B) whenever $c\leq b$. Moreover, using the same techniques from above we get the mixed strategy Nash equilibrium where both players play $A$ with probability $p=\frac{b-d}{a+b-c-d}$. This is a reasonable mixed strategy whenever $b\geq d$ and $a\geq c$ and at least one of the inequalities is strict. 
		
\section{Structured Coordination Games}
	%structured two strategy games
	%structures n strategy games
	%particular structure
	%general structure
	In order to move beyond the simple two player coordination games one must make decisions about how individuals interact. Letting this game play out on a network where nodes are players and each edge represents a pairwise coordination game can result in intriguing dynamics. 
	
	The coordination game in a structured population has been studied before \cite{Buskens2016, Tomassini2010} but these studies have only considered $n$ players picking from 2 strategies. We seek to extend this work by examining the situation in which there are up to $n$ discrete strategies available (which is trivially extendable to the case when there are an infinite number of strategies available)
	An other study looked at a game which could be called an $n\times n$ structured coordination game\cite{Szabo2016} but the structure was always a square lattice and all but two of the strategies were neutral. 
\section{Continuous Coordination Games}
	%anything about continuous strategies in coordination games
\section{applications to theoretical ecology}
	%Social dynamics
	%Invasability of cooperative behavior
\bibliographystyle{plain}
\bibliography{structuredcoord}
\end{document}
