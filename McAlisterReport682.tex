\documentclass[]{article}
\usepackage{amsmath}
\usepackage{amssymb}

%opening
\title{Python Tools for the Analysis of a Graph Theoretical Dynamical System}
\author{John McAlister}

\begin{document}

\maketitle

\begin{abstract}
Here I present a set of tools written in python using the \verb*|networkx| library used for the analysis and visualization of a graph theoretical dynamical system inspired by a game theoretical approach to homophily. Through an object oriented approach, I investigate the dynamical system as an initial value problem then try to consider how to answer questions about a boundary value problem.
\end{abstract}

\section{Homophily though a Graph Theoretical Dynamical System}
Homophily, which is observed as the tendency to be associated with those similar to ourselves can emerge by simple mimicry behavior wherein an individual conforms to the behavior which is most common among their associates. Although this type of problem is thoroughly studied for strategies which exist in a metric space, when strategies are not measurable against one another and offer truly neutral intrinsic fitnesses, we can no longer use the standard ODE or difference equation models. Additionally, when individuals don't interact equally with all group members this game must be thought of on a graph. 
\subsection{The Game}
	Consider a connected graph $G(V,E)$ and a (not necessarily proper) vertex coloring of that graph $\mathbf{u}$. Now imagine the game where the fitness of each vertex is determined as the number of its neighbors which share the same color as it. The fitness function is $w:V\times C\times U\rightarrow \mathbb{N}$
\subsection{The Dynamical System}
\subsection{Goals}
\section{Initial Value Problem}
\subsection{Preliminaries}
\subsection{Orbit Object}
\subsection{Understanding Cliques}
\subsection{Visualization}
\subsection{Enumerating Equilibria}
\subsection{Next Questions}
\section{Towards a Boundary Value Problem}
\subsection{Reframing the question}


\end{document}
